Interacting with SQLite Database

\hypertarget{sqlite-database}{%
\subsection{SQLite Database}\label{sqlite-database}}

This application helps the user to create database objects and interact
with them. The user can provide a directory on which a database object
can be based. This directory should contain image files (.jpg, .jpeg or
.png) and corresponding metadata files (.txt) carrying the same file
name. The images are stored in the database along with the corresponding
metadata.

\textbf{Columns}:

\begin{itemize}
\tightlist
\item
  AUHTOR (the author name)\\
\item
  TITLE (the title of the image)\\
\item
  LINK (the URL related to the image)\\
\item
  PICTURE (the image stored as a byte array)
\end{itemize}

It is possible to print the values of the created table. If the user
only wants the metadata of a specific sample, then the metadata can be
retrieved by specifying either the author or the title and the metadata
will be saved as a .txt file.

\hypertarget{getting-started}{%
\subsection{Getting Started}\label{getting-started}}

\hypertarget{description}{%
\subsubsection{Description}\label{description}}

Main Class: \textbf{CommandLineInterface.Java} Subsidiary Classes:
\textbf{Image.java}, \textbf{Database.Java}, \textbf{MetaDataAPI.java}

\hypertarget{installing}{%
\subsubsection{Installing}\label{installing}}

How to set up an working environment for this application: \#\#\#
Installing SQLite

Before trying to install, please check whether the installation has
already been made.

\begin{verbatim}
$ sqlite3
\end{verbatim}

If SQLite is already installed, you should get the following message:

\begin{verbatim}
SQLite version 3.29.0 2019-07-10 17:32:03
Enter ".help" for usage hints.
Connected to a transient in-memory database.
Use ".open FILENAME" to reopen on a persistent database.
sqlite>
\end{verbatim}

If not, please follow these instructions to install SQLite.

Go to the SQLite webpage (https://www.sqlite.org/download.html) and
download the most recent version of **SQLite-autoconf-*.tar.gz**. Type
the following commands into the command line to unzip and install the
package.

\begin{verbatim}
$ tar xvfz SQLite-autoconf-*.tar.gz

$ cd SQLite-autoconf-*

$ ./configure --prefix = /usr/local

$ make

$ make install
\end{verbatim}

Confirm successful installation by typing again

\begin{verbatim}
$ sqlite3
\end{verbatim}

\hypertarget{running-the-application}{%
\subsubsection{Running the Application}\label{running-the-application}}

In order to get the application running you will need to download the
repository from gitlab from the following link.

\begin{verbatim}
https://github.com/ShreyaKapoor18/biomed_images_metadb
\end{verbatim}

\begin{enumerate}
\def\labelenumi{\arabic{enumi}.}
\tightlist
\item
  Git clone the repository
\item
  Open Eclipse

  \begin{itemize}
  \tightlist
  \item
    Direct the workspace to biomed\_images\_metadb
  \item
    Import existing maven project
  \item
    Select the de.bit.pl02.task02 pom xml for importing the existing
    maven project.
  \item
    Do a maven install
  \end{itemize}
\item
  After the application is installed you will see a
  task02-0.0.1-SNAPSHOT.jar in the /target folder
\item
  To execute the jar use the following command (example in Commands
  file)

  \begin{itemize}
  \tightlist
  \item
    java -cp sqlite/target/task02-0.0.1-SNAPSHOT.jar
    de.bit.pl02.pp5.task02.CommandLineInterface 
  \item
    or use the Commands file to run from the base folder from your
    computer

    \begin{itemize}
    \tightlist
    \item
      bash SQLite/sqlite/Commands.txt \#\#\# Accessing a database
    \end{itemize}
  \end{itemize}
\end{enumerate}

For the program to be running you need to specify the name of the
database you want to make or the database you want to see that you have
already created. Therefore use the option

\textbf{-n} or \textbf{--name} \textbf{\emph{(Mandatory Option)}} :
Enter the path of the database and the name of the database you want to
make/see. Example(trialyy.db) database from PP5 repository in the
previous task has been made already.

\begin{verbatim}
-n your_path,your_database_name 
\end{verbatim}

Provide a directory that contains image files (.png, .jpeg, .jpg) and
corresponding metadata files (.txt). In our case it is the PP5 directory
created in task01.The metadata files have to have the following
information:

\begin{verbatim}
AUTHOR: xxx
TITLE: xxx
https://www.example.com (only for files that contain a link)
\end{verbatim}

Therefore use the option

\textbf{-d} or \textbf{--directory}: Enter the file directory from which
you want to store the images

\begin{verbatim}
-d your_directory_path
\end{verbatim}

\hypertarget{querying-a-database}{%
\subsubsection{Querying a database}\label{querying-a-database}}

Now you can choose how you want to query the database. You can either
retrieve metadata information or the image as a .jpg file.

\hypertarget{retrieval-of-metadata-information}{%
\paragraph{\texorpdfstring{\textbf{Retrieval of metadata
information}}{Retrieval of metadata information}}\label{retrieval-of-metadata-information}}

You can give information about the author or the title to retrieve
additional metadata. To query by the author use the option

\textbf{-gma} or \textbf{--getMetabyAuthor}: Enter the name of the
author of which you want to retrieve the metadata and the outputpath
where to save it at

\begin{verbatim}
-gma author_name,output_path
\end{verbatim}

To query by the title use the option

\textbf{-gmt} or \textbf{--getMetabyTitle}: Enter the name of the title
of which you want to retrieve the metadata and the outputpath where to
save it at

\begin{verbatim}
-gmt title_name,output_path
\end{verbatim}

\hypertarget{retrieval-of-images}{%
\paragraph{\texorpdfstring{\textbf{Retrieval of
images}}{Retrieval of images}}\label{retrieval-of-images}}

You can give information about the author or title to retrieve the
images. To query by the author use the option

\textbf{-gia} or \textbf{--getImagebyAuthor}: Enter the name of the
author from which you want the image and the outputpath where to save it
at

\begin{verbatim}
-gia author_name,output_path
\end{verbatim}

To query by the title use the option

\textbf{-git} or \textbf{--getImagebyTitle}: Enter the name of the title
from which you want the image and the outputpath where to save it at

\begin{verbatim}
-git title_name,output_path
\end{verbatim}

\hypertarget{prerequisites}{%
\subsubsection{Prerequisites}\label{prerequisites}}

The Java Version: 1.8.0\_231 is used for this application. Apache Maven
Version 3.6.3 was installed from https://maven.apache.org/download.cgi.
Therefore the binaries apache-maven-3.6.3-bin.zip were downloded.

The following dependencies were added to Maven:

\textbf{SQLite-JDBC} Version 3.18.0\\
\textbf{commons-cli} Version 1.4\\
\textbf{commons-io} Version 2.6\\
\textbf{commons-lang3} Version 3.4

by adding the according dependencies to the pom.xml file:

\begin{verbatim}
<dependency>
        <groupId>org.xerial</groupId>
        <artifactId>sqlite-jdbc</artifactId>
        <version>3.18.0</version>
</dependency>

<dependency>
    <groupId>commons-cli</groupId>
    <artifactId>commons-cli</artifactId>
    <version>1.4</version>
</dependency>

<dependency>
        <groupId>commons-io</groupId>
        <artifactId>commons-io</artifactId>
        <version>2.6</version>
 </dependency>

<dependency>
        <groupId>org.apache.commons</groupId>
        <artifactId>commons-lang3</artifactId>
    <version>3.4</version>
</dependency>
\end{verbatim}

\hypertarget{built-with}{%
\subsection{Built With}\label{built-with}}

\begin{itemize}
\tightlist
\item
  \href{https://maven.apache.org/}{Maven} - Dependency Management
\end{itemize}

\hypertarget{authors}{%
\subsection{Authors}\label{authors}}

\textbf{Shreya Kapoor}\\
\textbf{Sophia Krix}\\
\textbf{Gemma van der Voort}
